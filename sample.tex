\documentclass[10pt,a4paper,ragged2e,withhyper]{altacv}

\geometry{left=1.25cm,right=1.25cm,top=1.5cm,bottom=1.5cm,columnsep=1.2cm}

\usepackage{paracol}
\usepackage[vietnamese]{babel}
\usepackage{hyperref}

\iftutex 
  \setmainfont{Roboto Slab}
  \setsansfont{Lato}
  \renewcommand{\familydefault}{\sfdefault}
\else
  \usepackage[rm]{roboto}
  \usepackage[defaultsans]{lato}
  \renewcommand{\familydefault}{\sfdefault}
\fi

\definecolor{SlateGrey}{HTML}{2E2E2E}
\definecolor{LightGrey}{HTML}{666666}
\definecolor{DarkPastelRed}{HTML}{51715c}
\definecolor{PastelRed}{HTML}{51715c}
\definecolor{GoldenEarth}{HTML}{51715c}
\colorlet{name}{black}
\colorlet{tagline}{PastelRed}
\colorlet{heading}{DarkPastelRed}
\colorlet{headingrule}{GoldenEarth}
\colorlet{subheading}{PastelRed}
\colorlet{accent}{PastelRed}
\colorlet{emphasis}{SlateGrey}
\colorlet{body}{LightGrey}

\renewcommand{\namefont}{\Huge\rmfamily\bfseries}
\renewcommand{\personalinfofont}{\footnotesize}
\renewcommand{\cvsectionfont}{\LARGE\rmfamily\bfseries}
\renewcommand{\cvsubsectionfont}{\large\bfseries}


\renewcommand{\cvItemMarker}{{\small\textbullet}}
\renewcommand{\cvRatingMarker}{\faCircle}


\usepackage[backend=biber,style=numeric]{biblatex}
\addbibresource{public.bib}

\begin{document}
\name{Nguyễn Chí Bằng}
\tagline{Sinh viên năm 3, ngành Toán Ứng dụng, Đại học Sài Gòn}
\photoR{2.8cm}{chibang}

\personalinfo{%
  % Not all of these are required!
  % \email{chibangn1@gmail.com}
  \mailaddress{chibangn1@gmail.com}
  % \github{Bancie}
  \phone{+84 832 946 009}
  \location{TP.HCM}
  % \homepage{www.homepage.com}
  % \twitter{@twitterhandle}
  % \xtwitter{@x-handle}
  % \linkedin{your_id}
  % \orcid{0000-0000-0000-0000}
  %% You can add your own arbitrary detail with
  %% \printinfo{symbol}{detail}[optional hyperlink prefix]
  % \printinfo{\faPaw}{Hey ho!}[https://example.com/]

  %% Or you can declare your own field with
  %% \NewInfoFiled{fieldname}{symbol}[optional hyperlink prefix] and use it:
  % \NewInfoField{gitlab}{\faGitlab}[https://gitlab.com/]
  % \gitlab{your_id}
  %%
  %% For services and platforms like Mastodon where there isn't a
  %% straightforward relation between the user ID/nickname and the hyperlink,
  %% you can use \printinfo directly e.g.
  % \printinfo{\faMastodon}{@username@instace}[https://instance.url/@username]
  %% But if you absolutely want to create new dedicated info fields for
  %% such platforms, then use \NewInfoField* with a star:
  % \NewInfoField*{mastodon}{\faMastodon}
  %% then you can use \mastodon, with TWO arguments where the 2nd argument is
  %% the full hyperlink.
  % \mastodon{@username@instance}{https://instance.url/@username}
}

\makecvheader
%% Depending on your tastes, you may want to make fonts of itemize environments slightly smaller
% \AtBeginEnvironment{itemize}{\small}

%% Set the left/right column width ratio to 6:4.
\columnratio{0.6}

% Start a 2-column paracol. Both the left and right columns will automatically
% break across pages if things get too long.
\begin{paracol}{2}
\cvsection{Hoạt động dạy học}


\cvevent{Gia sư và Dạy kèm 1-1}{Lớp 8, lớp 9.}{2022 - 2024}{Online/Offline}
\begin{itemize}
\item Học viên đều đạt kết quả điểm số đề ra đầu khoá học.
\end{itemize}


\cvsection{Kinh nghiệm toán học}


\cvevent{Nghiên cứu và Phát triển phần mềm}{Tối ưu hoá Chuỗi thời gian}{7/2024 - Hiện tại}{Đại học Văn Lang}
\begin{itemize}
\item Nghiên cứu và phát triển thuật toán lập lịch tối ưu/tối ưu chuỗi thời gian cho đơn máy và đa máy.
\item Thiết kế thư viện Python, hỗ trợ triển khai và tối ưu hóa thuật toán lập lịch.
\end{itemize}

\divider

\cvevent{Nghiên cứu khoa học cấp trường}{Phương pháp xử lý bài toán Tối ưu tuyến tính nguyên}{8/2023 - 5/2024}{Đại học Sài Gòn}
\begin{itemize}
\item Nghiên cứu và ứng dụng thuật toán Branch and Bound và Gomory, giải quyết bài toán tối ưu tuyến tính với ràng buộc nghiệm nguyên.
\item Thiết kế thư viện Python, hỗ trợ triển khai và tối ưu hóa thuật toán Branch and Bound.
\item Đề tài nghiên cứu xếp loại xuất sắc (Điểm 90/100) , nằm trong top 30\% trong tổng 179 đề tài cấp trường.
\end{itemize}


\cvsection{Dự án}

\cvevent{TiLearn}{A Python library for machine scheduling.}{7/2024 - Hiện tại}{}
\begin{itemize}
\item Ứng dụng thuật toán lập lịch và tối ưu chuỗi thời gian.
\item Nền tảng mã nguồn mở cung cấp công cụ và tài nguyên giúp cá nhân và đội nhóm quản lý thời gian.
\item \href{https://pypi.org/project/TiLearn/}{PyPI} \faPython
\item \href{https://github.com/Bancie/TiLearn}{GitHub} \faGithubSquare
\end{itemize}

% \divider

% \cvevent{Optimization-Oracle}{An open-source library for efficient MILP algorithms.}{8/2023 - Hiện tại}{}
% \begin{itemize}
% \item Thư viện mã nguồn mở phát triển và cải thiện thuật toán Branch and Bound giúp xử lý bài toán Tối ưu tuyến tính nguyên hỗn hợp (MILP).
% \item \href{https://github.com/Bancie/Optimization-Oracle}{GitHub} \faGithubSquare
% \end{itemize}

\medskip

%\cvsection{A Day of My Life}

% Adapted from @Jake's answer from http://tex.stackexchange.com/a/82729/226
% \wheelchart{outer radius}{inner radius}{
% comma-separated list of value/text width/color/detail}
%\wheelchart{1.5cm}{0.5cm}{%
 % 6/8em/accent!30/{Sleep,\\beautiful sleep},
 % 3/8em/accent!40/Hopeful novelist by night,
 % 8/8em/accent!60/Daytime job,
 % 2/10em/accent/Sports and relaxation,
 % 5/6em/accent!20/Spending time with family
%}

% use ONLY \newpage if you want to force a page break for
% ONLY the current column
%\newpage

\cvsection{Đề tài nghiên cứu}

\nocite{*}

\printbibliography[heading=none]



% \bibliographystyle{plain}
% \bibliography{public}

% \printbibliography[heading=pubtype,title={\printinfo{\faBook}{Books}},type=book]
% \printbibliography[type=book,title={Books only}]

%\divider

% \printbibliography[heading=pubtype,title={\printinfo{\faFile*[regular]}{Articles}},type=article]

% \printbibliography[heading=subbibliography, title={Publications}, type=article]

%\divider

%\printbibliography[heading=pubtype,title={\printinfo{\faUsers}{Conference Proceedings}},type=inproceedings]

%% Switch to the right column. This will now automatically move to the second
%% page if the content is too long.
\switchcolumn

%\cvsection{My Life Philosophy}

%\begin{quote}
%``Something smart or heartfelt, preferably in one sentence.''
%\end{quote}

\cvsection{Thành tích}



\cvachievement{\faAward}{5/2024}{Xếp loại Xuất xắc (Top 30\%) Đề tài Nghiên cứu khoa học cấp trường.}

\divider

\cvachievement{\faAward}{7/2021}{8.8 ĐIỂM MÔN TOÁN - THPTQG 2021.}


%\divider

%\cvachievement{\faHeartbeat}{Another achievement}{more details about it of course}

% \cvsection{Thế mạnh}

% Don't overuse these \cvtag boxes — they're just eye-candies and not essential. If something doesn't fit on a single line, it probably works better as part of an itemized list (probably inlined itemized list), or just as a comma-separated list of strengths.

% The `ragged2e` document class option might cause automatic linebreaks between \cvtag to fail.
% Either remove the ragged2e option; or 
% add \LaTeXraggedright in the paragraph for these \cvtag
%{\LaTeXraggedright
%\cvtag{Hard-working}
%\cvtag{Eye for detail}
%\cvtag{Motivator \& Leader}
%\par}

%\divider\smallskip

%% ...Or manually add linebreaks yourself
% \cvtag{C/C++}
% \cvtag{Python}
% \cvtag{Data Analysis}\\
% \cvtag{Library Development}


% \divider\smallskip

% \cvtag{Optimization}
% \cvtag{Operations Research}\\
% \cvtag{Machine Learning}
% \cvtag{Deep Learning}

% \cvsection{Kỹ năng}

% \cvskill{Maths \& Code}{4}
% \divider

% \cvskill{English}{4}

% \cvskill{German}{3.5} %% Supports X.5 values.

%% Yeah I didn't spend too much time making all the
%% spacing consistent... sorry. Use \smallskip, \medskip,
%% \bigskip, \vspace etc to make adjustments.
\medskip

\cvsection{Học vấn}

\cvevent{Năm 3.\ Ngành Toán Ứng Dụng}{Đại học Sài Gòn}{2022 -- 2026}{}
\begin{itemize}
\item GPA: 3.02/4.00
\end{itemize}

\divider

\cvevent{Học sinh lớp chuyên Vật Lý}{Trường THPT Chuyên Vị Thanh}{2018 -- 2021}{}
% Thesis title: Wonderful Research

% \divider

% \cvevent{M.Sc.\ in Your Discipline}{Your University}{Sept 2001 -- June 2002}{}

% \divider

% \cvevent{B.Sc.\ in Your Discipline}{Stanford University}{Sept 1998 -- June 2001}{}

% \divider

\cvsection{Minh chứng}

% \cvref{name}{email}{mailing address}
\cvref{PGS.TS.\ Tạ Quang Sơn}{Khoa Toán - Ứng dụng}{taquangson@sgu.edu.vn}
{Đại học Sài Gòn, TP.HCM, Việt Nam}

\divider

\cvref{TS.\ Lê Minh Huy}{Khoa Khoa học cơ bản}{huy.lm@vlu.edu.vn}
{Đại học Văn Lang, TP.HCM, Việt Nam}

\divider

\cvref{TS.\ Trần Thanh Hiệp}{Khoa Khoa học Ứng dụng}{ttthiep.sdh241@hcmut.edu.vn}
{Đại học Bách khoa TP.HCM, Việt Nam}


\end{paracol}


\end{document}
