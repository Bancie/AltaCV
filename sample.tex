\documentclass[10pt,a4paper,ragged2e,withhyper]{altacv}

\geometry{left=1.25cm,right=1.25cm,top=1.5cm,bottom=1.5cm,columnsep=1.2cm}

\usepackage{paracol}
% \usepackage[vietnamese]{babel}
\usepackage{hyperref}

\iftutex 
  \setmainfont{Roboto Slab}
  \setsansfont{Lato}
  \renewcommand{\familydefault}{\sfdefault}
\else
  \usepackage[rm]{roboto}
  \usepackage[defaultsans]{lato}
  \renewcommand{\familydefault}{\sfdefault}
\fi

\definecolor{SlateGrey}{HTML}{2E2E2E}
\definecolor{LightGrey}{HTML}{666666}
\definecolor{DarkPastelRed}{HTML}{51715c}
\definecolor{PastelRed}{HTML}{51715c}
\definecolor{GoldenEarth}{HTML}{51715c}
\colorlet{name}{black}
\colorlet{tagline}{PastelRed}
\colorlet{heading}{DarkPastelRed}
\colorlet{headingrule}{GoldenEarth}
\colorlet{subheading}{PastelRed}
\colorlet{accent}{PastelRed}
\colorlet{emphasis}{SlateGrey}
\colorlet{body}{LightGrey}

\renewcommand{\namefont}{\Huge\rmfamily\bfseries}
\renewcommand{\personalinfofont}{\footnotesize}
\renewcommand{\cvsectionfont}{\LARGE\rmfamily\bfseries}
\renewcommand{\cvsubsectionfont}{\large\bfseries}


\renewcommand{\cvItemMarker}{{\small\textbullet}}
\renewcommand{\cvRatingMarker}{\faCircle}


\usepackage[backend=biber,style=numeric]{biblatex}
\addbibresource{public.bib}

\begin{document}
\name{NGUYEN CHI BANG}
\tagline{Lead developer \& Researcher}
\photoR{2.8cm}{bancie}

\personalinfo{%
  % Not all of these are required!
  % \email{chibangn1@gmail.com}
  \mailaddress{chibangn1@gmail.com}
  \github{Bancie}
  \phone{+84 832 946 009}
  \location{HCM City, VIETNAM}
  % \homepage{www.homepage.com}
  % \twitter{@twitterhandle}
  % \xtwitter{@x-handle}
  % \linkedin{your_id}
  % \orcid{0000-0000-0000-0000}
  %% You can add your own arbitrary detail with
  %% \printinfo{symbol}{detail}[optional hyperlink prefix]
  % \printinfo{\faPaw}{Hey ho!}[https://example.com/]

  %% Or you can declare your own field with
  %% \NewInfoFiled{fieldname}{symbol}[optional hyperlink prefix] and use it:
  % \NewInfoField{gitlab}{\faGitlab}[https://gitlab.com/]
  % \gitlab{your_id}
  %%
  %% For services and platforms like Mastodon where there isn't a
  %% straightforward relation between the user ID/nickname and the hyperlink,
  %% you can use \printinfo directly e.g.
  % \printinfo{\faMastodon}{@username@instace}[https://instance.url/@username]
  %% But if you absolutely want to create new dedicated info fields for
  %% such platforms, then use \NewInfoField* with a star:
  % \NewInfoField*{mastodon}{\faMastodon}
  %% then you can use \mastodon, with TWO arguments where the 2nd argument is
  %% the full hyperlink.
  % \mastodon{@username@instance}{https://instance.url/@username}
}

\makecvheader
%% Depending on your tastes, you may want to make fonts of itemize environments slightly smaller
% \AtBeginEnvironment{itemize}{\small}

%% Set the left/right column width ratio to 6:4.
\columnratio{0.6}

% Start a 2-column paracol. Both the left and right columns will automatically
% break across pages if things get too long.
\begin{paracol}{2}
\cvsection{Experience}


\cvevent{Research and Software Development}{Time Sequence Optimization}{7/2024 - Ongoing}{HCMUT}
\begin{itemize}
\item Research and development of optimal scheduling algorithms, integrating machine learning and deep learning for single-machine and multi-machine systems.
\item Design of a Python library to support the implementation and optimization of scheduling algorithms.
\end{itemize}


\divider


\cvevent{Operations Research}{Methods for Solving Integer Linear Optimization Problems}{8/2023 - 5/2024}{SGU}
\begin{itemize}
\item Researched and applied Branch and Bound and Gomory algorithms to solve linear optimization problems with integer constraints.
\item Designed a Python library to support the implementation and optimization of the Branch and Bound algorithm.
\item Research project rated as excellent, ranked in the top 30\% out of a total of 179 projects.
\end{itemize}


\cvsection{Projects}

\cvevent{TiLearn}{A Python library for machine scheduling.}{7/2024 - Ongoing}{}
\begin{itemize}
\item Applied scheduling algorithms and machine learning methods.
\item Open-source platform providing tools and resources to help individuals and teams manage time effectively.
\item \href{https://pypi.org/project/TiLearn/}{PyPI} \faPython
\item \href{https://github.com/Bancie/TiLearn}{GitHub} \faGithubSquare
\end{itemize}

\divider

\cvevent{Optimization-Oracle}{An open-source library for efficient MILP algorithms.}{8/2023 - Ongoing}{}
\begin{itemize}
\item Open-source library for developing and improving Branch and Bound algorithms to handle Mixed Integer Linear Programming (MILP) problems.
\item \href{https://github.com/Bancie/Optimization-Oracle}{GitHub} \faGithubSquare
\end{itemize}

\medskip

%\cvsection{A Day of My Life}

% Adapted from @Jake's answer from http://tex.stackexchange.com/a/82729/226
% \wheelchart{outer radius}{inner radius}{
% comma-separated list of value/text width/color/detail}
%\wheelchart{1.5cm}{0.5cm}{%
 % 6/8em/accent!30/{Sleep,\\beautiful sleep},
 % 3/8em/accent!40/Hopeful novelist by night,
 % 8/8em/accent!60/Daytime job,
 % 2/10em/accent/Sports and relaxation,
 % 5/6em/accent!20/Spending time with family
%}

% use ONLY \newpage if you want to force a page break for
% ONLY the current column
%\newpage

\cvsection{Publications}

\nocite{*}

\printbibliography[heading=none]



% \bibliographystyle{plain}
% \bibliography{public}

% \printbibliography[heading=pubtype,title={\printinfo{\faBook}{Books}},type=book]
% \printbibliography[type=book,title={Books only}]

%\divider

% \printbibliography[heading=pubtype,title={\printinfo{\faFile*[regular]}{Articles}},type=article]

% \printbibliography[heading=subbibliography, title={Publications}, type=article]

%\divider

%\printbibliography[heading=pubtype,title={\printinfo{\faUsers}{Conference Proceedings}},type=inproceedings]

%% Switch to the right column. This will now automatically move to the second
%% page if the content is too long.
\switchcolumn

%\cvsection{My Life Philosophy}

%\begin{quote}
%``Something smart or heartfelt, preferably in one sentence.''
%\end{quote}

\cvsection{Most proud of}

\cvachievement{\faHeartbeat}{Burning Passion}{Always passionate about the work being done and constantly learning new things.}

\divider

\cvachievement{\faAward}{Chill Guy}{Knows how to turn challenges and pressure into motivation.}

%\divider

%\cvachievement{\faHeartbeat}{Another achievement}{more details about it of course}

\cvsection{Strengths}

% Don't overuse these \cvtag boxes — they're just eye-candies and not essential. If something doesn't fit on a single line, it probably works better as part of an itemized list (probably inlined itemized list), or just as a comma-separated list of strengths.

% The `ragged2e` document class option might cause automatic linebreaks between \cvtag to fail.
% Either remove the ragged2e option; or 
% add \LaTeXraggedright in the paragraph for these \cvtag
%{\LaTeXraggedright
%\cvtag{Hard-working}
%\cvtag{Eye for detail}
%\cvtag{Motivator \& Leader}
%\par}

%\divider\smallskip

%% ...Or manually add linebreaks yourself
\cvtag{C/C++}
\cvtag{Python}
\cvtag{Data Analysis}\\
\cvtag{Library Development}


\divider\smallskip

\cvtag{Optimization}
\cvtag{Operations Research}\\
\cvtag{Machine Learning}
\cvtag{Deep Learning}

\cvsection{Skills}

\cvskill{Maths \& Code}{4.5}
\divider

\cvskill{English}{4}

% \cvskill{German}{3.5} %% Supports X.5 values.

%% Yeah I didn't spend too much time making all the
%% spacing consistent... sorry. Use \smallskip, \medskip,
%% \bigskip, \vspace etc to make adjustments.
\medskip

\cvsection{Education}

\cvevent{JR.\ in Applied Mathematics}{Saigon University}{2022 -- 2026}{}
% Thesis title: Wonderful Research

% \divider

% \cvevent{M.Sc.\ in Your Discipline}{Your University}{Sept 2001 -- June 2002}{}

% \divider

% \cvevent{B.Sc.\ in Your Discipline}{Stanford University}{Sept 1998 -- June 2001}{}

% \divider

\cvsection{Referees}

% \cvref{PGS.TS.\ Tạ Quang Sơn}{Khoa Toán - Ứng dụng}{taquangson@sgu.edu.vn}
% {Đại học Sài Gòn, TP.HCM, Việt Nam}

% \divider

% \cvref{TS.\ Lê Minh Huy}{Khoa Khoa học cơ bản}{huy.lm@vlu.edu.vn}
% {Đại học Văn Lang, TP.HCM, Việt Nam}

% \divider

% \cvref{TS.\ Trần Thanh Hiệp}{Khoa Khoa học Ứng dụng}{ttthiep.sdh241@hcmut.edu.vn}
% {Đại học Bách khoa TP.HCM, Việt Nam}

\cvref{Assoc. Prof. Dr.\ Ta Quang Son}{Faculty of Applied Mathematics \\\& Applications}{taquangson@sgu.edu.vn}
{SGU, HCM City, Vietnam}

\divider

\cvref{Dr.\ Le Minh Huy}{Faculty of Fundamental Sciences}{huy.lm@vlu.edu.vn}
{VLU, HCM City, Vietnam}

\divider

\cvref{Dr.\ Tran Thanh Hiep}{Faculty of Applied Sciences}{ttthiep.sdh241@hcmut.edu.vn}
{HCMUT, HCM City, Vietnam}


\end{paracol}


\end{document}
